\documentclass{scrreprt}
\usepackage{paralist}
\usepackage{graphicx}
\usepackage{hcar}

\begin{document}

\begin{hcarentry}{Yahc}
\report{Miguel Pagano}
\status{experimental, maintained}
\participants{Renato Cherini}% optional
\makeheader


The first course on algorithms in CS at {\it Universidad Nacional de
  C\'ordoba} is centered on the derivations of algorithms from
specifications, as proposed by Bird\cite{bird},
Dijkstra\cite{dijkstra} and Hoogerwoord\cite{hoo}. To achieve this
goal, students should acquire the ability to manipulate complex
predicate formulae; thus the students first learn how to prove
theorems in a propositional calculus similar to the equational
propositional logic of Gries and Schneier\cite{gries}.

During the semester students make many derivations as exercises and it
is helpful for them to have a tool for checking the correctness of
their solutions. Yahc checks the correctness of a sequence of
applications of some axioms and theorems to the formulae students are
trying to prove. The student starts a derivation by entering an
initial formula and a goal and then proceeds by telling Yahc which
axiom will be used and the expected outcome of applying the axiom as a
rewrite rule; if that rewriting step is correct then the process
continues until the student reaches the goal.

At this moment the tool is in an early stage and we only consider
propositional connectives (some of them associative-commutative).  We
expect to extend Yahc for allowing the resolution of logical puzzles.
In the long term we are planning to consider an equational calculus
with functions defined by induction over lists and natural numbers.

\FurtherReading
  \url{http://www.cs.famaf.unc.edu.ar/~mpagano/yahc/}
\end{hcarentry}

\begin{thebibliography}{99}
\bibitem{bird} Richard Bird, {\it Introduction to functional
    programming using Haskell}.  Prentice Hall series in computer
  science, 1998.
\bibitem{dijkstra} Edsger W. Dijkstra {\it A Discipline of
    Programming}.  Prentice Hall, 1976.
\bibitem{hoo} Rob R. Hoogerwoord {\it The design of functional
    programs: a calculational approach}. Eindhoven: Technische
  Universiteit Eindhoven, 1989.
\bibitem{gries} David Gries and Fred B. Schneider, {\it A Logical
    Approach to Discrete Math}. Springer Verlag, 1993.
\end{thebibliography}

\end{document}
